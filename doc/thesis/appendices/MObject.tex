% !TEX root = ../thesis.tex

%%%%%%%%%%%%%%%%%%%%%%%%%%%%%%%%%%%%%%%%%%%%%%%%%%%%%%%%%%%%%%%
% Chapter: MObject
%%%%%%%%%%%%%%%%%%%%%%%%%%%%%%%%%%%%%%%%%%%%%%%%%%%%%%%%%%%%%%%
\chapter{The MObject class}\label{apdx:MObject}
This class is the managed data interpreter; therefore, it first it setups the fields of a managed object based on the schema klass.
Next, it handles all the calls to the managed objects' methods and finally is responsible for type checking of the fields' values during initialization or assignment.

The usage of the \texttt{schemaKlass} for setting up the fields is shown in Listing \ref{lst:setup_fields}.

\begin{sourcecode}
	\begin{lstlisting}[language=Java, escapechar=|]
public MObject(Klass schemaKlass, Object... initializers) {
	this.schemaKlass = schemaKlass;
	this.schemaKlass.fields().forEach(this::safeSetupField);|\label{line:setup_fields}|
	this.safeInitializeProps(initializers);
}

protected void setupField(Field field) {
	if (!field.many()) { |\label{line:setup_field_many_check}|
		if (field.type().schemaKlass().name().equals("Primitive")) {|\label{line:instanceof}|
			this.props.put(field.name(), new MObjectFieldSinglePrimitive(this, field));
		} else {
			this.props.put(field.name(), new MObjectFieldSingleMObj(this, field));
		}
	} else {
		if (field.type().schemaKlass().name().equals("Primitive")) {
			this.props.put(field.name(), new MObjectFieldManyList(this, field));
		} else {

			Klass klassType = (Klass) field.type();
			if (klassType.key()!= null) {
				this.props.put(field.name(), new MObjectFieldManySet(this, field));
			} else {
				this.props.put(field.name(), new MObjectFieldManyList(this, field));
			}
		}
	}
}
	\end{lstlisting}
	\caption{MObject: setup fields}
	\label{lst:setup_fields}
\end{sourcecode}

The \texttt{schemaKlass} is given to the \texttt{MObject} by the \texttt{DataManager} that is responsible for creating it.
Using this \texttt{schemaKlass} the \texttt{MObject} setups the \texttt{Fields} of the Klass, Line \ref{line:setup_fields}.
Inside the \texttt{setupField} method the interpretation of the schema is performed.
In particular, in Line \ref{line:setup_field_many_check} we check if that field is a multi-value field, and if not, we just set it up as a \texttt{Primitive} or a \texttt{Klass} accordingly. 
Consider that the \texttt{field.type().schemaKlass().name()} is used like a common \texttt{instanceof} in Line \ref{line:instanceof}.
In case the field has many values, we first check if it is \texttt{Primitive}, since we do not support Set of \texttt{Primitive}s.
Following that, we check if a \texttt{Key} field exists on that field's type and in that case this field is a Set, otherwise it is a List.

The invocation handling process of managed object is showed in Listing \ref{lst:mobject_invocation_handler}.

\begin{sourcecode}
	\begin{lstlisting}[language=Java, escapechar=|]
public Object invoke(
	Object proxy, Method method, Object[] args) throws Throwable {
	final String fieldName = method.getName();

	if (method.isDefault()) { // if the method is default, invoke this one
		return _callDefaultMethod(proxy, method, args);
	}

	// This is a way to execute the "attached" methods of the derived Managed Objects,
	for (Method declaredMethod : this.getClass().getMethods()) {
		if (declaredMethod.getName().equals(fieldName)) {
			return method.invoke(this, args);
		}
	}

	// Managed Object
	MObjectField mObjectField = this.props.get(fieldName);
	boolean isMany = mObjectField.getField().many();

	if (args == null) {
		return _get(fieldName); // return the field's value
	}

	boolean isAssignment = false;
	Object fieldArgs = args[0];

	if (fieldArgs.getClass().isArray() && ((Object [])fieldArgs).length > 0) {
		isAssignment = true;
	}

	if (isAssignment) {
		if (((Object [])fieldArgs).length == 1 && !isMany) {
			_set(fieldName, ((Object [])fieldArgs)[0]);
		} else {
			_set(fieldName, fieldArgs);
		}
		return null;
	}
	return _get(fieldName);
}
	\end{lstlisting}
	\caption{MObject: invocation handler}
	\label{lst:mobject_invocation_handler}
\end{sourcecode}

The type checking for each field is performed by the classes \texttt{MObjectFieldSinglePrimitive}, \texttt{MObjectFieldSingleMObj}, \texttt{MObjectFieldManyList} and \texttt{MObjectFieldManySet}.

The basic structure of such class is given in Listing \ref{lst:MObjectField}.

\begin{sourcecode}
	\begin{lstlisting}[language=Java, escapechar=|]
// A field of managed data
public abstract class MObjectField {

	// the owner of the field as an Managed Object.
	protected final MObject owner;

	// the Field.
	protected final Field field;

	// the Inverse of the field.
	protected final Field inverse;

	public MObjectField(MObject owner, Field field) {
		this.owner = owner;
		this.field = field;
		this.inverse = field.inverse();
	}

	// Initializes the field with a value
	public abstract void init(Object value) 
		throws InvalidFieldValueException, NoKeyFieldException;

	// Checks the given value if it is valid
	protected abstract void check(Object value) 
		throws InvalidFieldValueException;

	// Returns a default value for this kind of field.
	protected abstract Object defaultValue() 
		throws UnknownTypeException;

	// Sets a value to the field.
	public abstract void set(Object value) 
		throws InvalidFieldValueException, NoKeyFieldException;

	// Returns the value of the field
	public abstract Object get();

	// Returns the Field object that is wrapped.
	public Field getField() {
		return this.field;
	}
}
\end{lstlisting}
	\caption{MObjectField abstract class}
	\label{lst:MObjectField}
\end{sourcecode}