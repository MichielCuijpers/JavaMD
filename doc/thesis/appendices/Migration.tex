% !TEX root = ../thesis.tex

%%%%%%%%%%%%%%%%%%%%%%%%%%%%%%%%%%%%%%%%%%%%%%%%%%%%%%%%%%%%%
% Chapter: Migration
%%%%%%%%%%%%%%%%%%%%%%%%%%%%%%%%%%%%%%%%%%%%%%%%%%%%%%%%%%%%%
\chapter{JHotDraw Migration Process}\label{Migration Process}

\section{DrawingView}
One of the main components of JHotDraw is the \textit{DrawingView} interface.
As Figure \ref{fig:JHotDraw_DrawingView} illustrates, the \textit{DrawingView} is responsible for rendering \texttt{Drawings} and listening to its changes.
Additionally, it is responsible for receiving the user input and delegating it to the current tool.

\begin{figure}[H]
	\centering
  	\fbox{\includegraphics[width=.8\textwidth]{figures/JHotDraw_DrawingView.png}}
  	\caption{DrawingView of JHotDraw}
  	\label{fig:JHotDraw_DrawingView}
\end{figure}

Conclusively, \texttt{DrawingView} makes a good candidate for managed data migration.
The reason is that the specifications of that class can be implemented in data managers and dynamically added to it.

\section{Managed Data DrawingView}
To support sub-typing on the \texttt{DrawingView} interface, we have implemented the \texttt{MDDrawingView}, namely Managed Data DrawingView, which replaced the \texttt{DrawingView} in JHotDraw.
Having this interface for super type, we still needed the actual managed data schemas.
As Figure \ref{fig:JHotDraw_DrawingView} shows, there are two implementations of the \texttt{DrawingView}.
In particular, the \texttt{StandardDrawingView}, which is the implementation that is used when a new drawing view is created in the application and the \texttt{NullDrawingView}, which represents a null drawing view as for the \textit{null-object} pattern.

Following their original design, we have implemented two schemas, one for the \texttt{StandardDrawingView} and one for the \texttt{NullDrawingView}, namely \texttt{MDStandardDrawingView} and \texttt{MDNullDrawingView} respectively.
The instances of those schemas have been used in the same way their counterparts are used in JHotDraw.
A snippet of the \texttt{MDStandardDrawingView} is shown in Listing \ref{lst:MDStandardDrawingView schema} \footnote{Most of the implementation has been omitted for brevity.}.

\begin{sourcecode}[H]
	\begin{lstlisting}[language=Java, escapechar=|]
public interface MDStandardDrawingView extends M, MDDrawingView { |\label{line:MDStandardDrawingView extends M, MDDrawingView}|
	...
	// Composition over inheritance, the original inherits the JPanel
	JPanel panel(JPanel... panel); |\label{line:jpanel composition}|

	default JPanel getPanel() { |\label{line:jpanel getter}|
	    return panel();
	}

	default void setPanel(JPanel _panel) {|\label{line:jpanel setter}|
	    panel(_panel);
	}
	...
	Rectangle damage(Rectangle... damage);
	Drawing drawing(Drawing... drawing);
	...
	default FigureEnumeration selectionZOrdered() { |\label{line:selectionZOrdered}|
		List result = CollectionsFactory.current().createList(selectionCount());
		FigureEnumeration figures = drawing().figures();

		while (figures.hasNextFigure()) {
			Figure f= figures.nextFigure();
			if (isFigureSelected(f)) {
				result.add(f);
			}
		}
		return new ReverseFigureEnumerator(result);
	}
	...
	default void repairDamage() { |\label{line:repairDamage}|
		if (getDamage() != null) {
			panel().repaint(damage().x, damage().y, damage().width, damage().height);
			setDamage(null);
		}
	}
	...
}
	\end{lstlisting}
	\caption{MDStandardDrawingView schema}
	\label{lst:MDStandardDrawingView schema}
\end{sourcecode}

Listing \ref{lst:MDStandardDrawingView schema} shows that the \texttt{MDStandardDrawingView} interface extends both \texttt{M} interface, defining that this is a schema definition, and \texttt{MDDrawingView}, for sub-type support.
Additionally, all the functionalities implemented in methods of the original \texttt{DrawingView}, in managed data they are implemented in default methods of the schema interface.
The fields of a schema can provide those methods with the managed object's current state.
As Lines \ref{line:selectionZOrdered} and \ref{line:repairDamage} show, the fields of the schema can be used to query their values inside the default methods.
Note that the code in the default methods is identical to the original \texttt{DrawingView}.
Furthermore, for consistency with the legacy code, we have implemented setters and getters, Lines \ref{line:jpanel setter} and \ref{line:jpanel getter}, for field values accessors.
This way we maintained consistency across in accessing values of the managed object.

A notable issue is that the original \texttt{StandardDrawingView} extends the \texttt{javax.swing.jpanel} class as Figure \ref{fig:JHotDraw_DrawingView} shows.
However, such a structure is not supported in managed data. 
Schema definitions can not extend classes.
To overcome this issue we defined the \texttt{JPanel} as a field in the schema, namely \textit{panel}.
To support the \texttt{JPanel} as a type of a field though, it is needs ti be defined as managed data.
By all means, the same holds for the remaining fields, such as \texttt{Rectangle} and \texttt{Drawing}.

As explained in Section \ref{Primitives Definition}, our framework provides external primitives definition by inheriting the \texttt{Primitives} interface.
The JHotDraw primitives definition is shown in Listing \ref{lst:JHotDraw Primitives Definition}.

\begin{sourcecode}[H]
	\begin{lstlisting}[language=Java, escapechar=|]
public interface JHotDrawPrimitives extends Primitives {
	javax.swing.JPanel JPanel();

	java.awt.Color Color();
	java.awt.Cursor Cursor();
	java.awt.Point Point();
	java.awt.Dimension Dimension();
	java.awt.Rectangle Rectangle();

	CH.ifa.draw.framework.DrawingEditor DrawingEditor();
	CH.ifa.draw.framework.Drawing Drawing();
	CH.ifa.draw.framework.Painter Painter();
	CH.ifa.draw.framework.PointConstrainer PointConstrainer();

	...
}
	\end{lstlisting}
	\caption{JHotDraw Primitives Definition}
	\label{lst:JHotDraw Primitives Definition}
\end{sourcecode}

This has been proven very helpful since we did not need to re-implement every field as managed data during the refactoring. 
Especially, classes that are provided by libraries such as \texttt{javax.swing} and \texttt{java.awt}.

\section{MDDrawingView Schema Factories}
In order to create instances of the defined \texttt{MDStandardDrawingView} and \texttt{MDNullDrawingView} schemas, we needed their factories.
Besides the schema factories, which is as simple as Listing \ref{lst:DrawingViewSchemaFactory} shows, we still needed a way to give initialization values to the schema instances the same way that the original \texttt{StandardDrawingView} does during construction.
Additionally, this factory should be used like Java's \texttt{new} keyword in the source code.
This factory just replicates the original \texttt{StandardDrawingView} constructor and is used from the program to create new instances of the schemas.
The code of the \texttt{MDStandardDrawingView} \textit{instances factory} is illustrated in Listing \ref{lst:MDStandardDrawingView Instances Factory}, in comparison to the original constructor, illustrated in Listing \ref{lst:DrawingView Constructor}.

\begin{sourcecode}[H]
	\begin{lstlisting}[language=Java, escapechar=|]
public interface DrawingViewSchemaFactory extends IFactory {
	MDStandardDrawingView DrawingView();
	MDNullDrawingView NullDrawingView();
}
	\end{lstlisting}
	\caption{DrawingView Schema Factory}
	\label{lst:DrawingViewSchemaFactory}
\end{sourcecode}

\begin{sourcecode}[H]
	\begin{lstlisting}[language=Java, escapechar=|]
public StandardDrawingView(DrawingEditor editor, int width, int height) {
	setAutoscrolls(true);
	fEditor = editor;
	fViewSize = new Dimension(width,height);
	setSize(width, height);
	fSelectionListeners = CollectionsFactory.current().createList();
	addFigureSelectionListener(editor()); |\label{line:addFigureSelectionListener_contructor}|
	setLastClick(new Point(0, 0));
	fConstrainer = null;
	fSelection = CollectionsFactory.current().createList();
	setDisplayUpdate(createDisplayUpdate());
	setBackground(Color.lightGray);
	addMouseListener(createMouseListener());
	addMouseMotionListener(createMouseMotionListener());
	addKeyListener(createKeyListener());
}
	\end{lstlisting}
	\label{lst:DrawingView Constructor}
	\caption{Original StandardDrawingView Constructor}
\end{sourcecode}

\begin{sourcecode}[H]
	\begin{lstlisting}[language=Java, escapechar=|]
public static MDDrawingView newDrawingView(
	DrawingEditor editor, int width, int height) {
	final MDStandardDrawingView drawingView = drawingViewSchemaFactory.DrawingView();
	MyJPanel jPanel = new MyJPanel();
	jPanel.setAutoscrolls(true);
 	jPanel.setSize(width, height);
	jPanel.setBackground(Color.lightGray);
	drawingView.panel(jPanel);
	jPanel.setDrawingView(drawingView);

	drawingView.editor(editor);
	drawingView.size(new Dimension(width, height));
	jPanel.setSize(width, height);
	drawingView.lastClick(new Point(0, 0));
	drawingView.constrainer(null);
	drawingView.setDisplayUpdate(new SimpleUpdateStrategy());
	drawingView.setBackground(Color.lightGray);
	drawingView.drawing(new StandardDrawing());

	jPanel.addMouseListener(...);
	jPanel.addMouseMotionListener(...);
	jPanel.addKeyListener(...);
	return drawingView;
}
	\end{lstlisting}
	\label{lst:MDStandardDrawingView Instances Factory}
	\caption{MDStandardDrawingView Instances Factory}
\end{sourcecode}

\section{MDDrawingView Integration}
Finally, in order to integrate the \texttt{MDDrawingView} managed objects in the existing system, first we had to replace every instance of \texttt{DrawingView} with \texttt{MDDrawingView}, every \texttt{StandardDrawingView} with \texttt{MDStandardDrawingView} and every \texttt{NullDrawingView} with \texttt{MDNullDrawingView} accordingly.
In addition, everywhere a new instance of these is created, we replaced it with our \textit{instances factory}.

For instance Listings \ref{lst:original_createDrawingView} and \ref{lst:refactored_createDrawingView} show how the code has been changed in the \texttt{DrawApplication} class.

\lstdefinestyle{smallJava}{
  basicstyle={\scriptsize\ttfamily},
  language=Java
}

\noindent\begin{minipage}{.45\textwidth}
\begin{lstlisting}[
	style=smallJava,
	caption=Original createDrawingView,
	frame=tlrb,
	label=lst:original_createDrawingView
]
Dimension d = getDrawingViewSize();

DrawingView newDrawingView = 
	new StandardDrawingView(this, d.width, d.height);

newDrawingView.setDrawing(newDrawing);
\end{lstlisting}
\end{minipage}\hfill
\begin{minipage}{.47\textwidth}
\begin{lstlisting}[
	style=smallJava,
	caption=ManagedData createDrawingView,
	frame=tlrb,
	label=lst:refactored_createDrawingView
]
Dimension d = getDrawingViewSize();

MDDrawingView newDrawingView = 
	MDDrawingViewFactory
		.newSubjectRoleDrawingView(this, d.width, d.height);

newDrawingView.setDrawing(newDrawing);
	\end{lstlisting}
\label{lst:createDrawingView}
\end{minipage}

The factory's code of the \texttt{MDDrawingView} with the observable data manager is shown in Listing \ref{lst:ManagedDataJHotDraw_MDDrawingView_Factory}.

\begin{sourcecode} [H]
	\begin{lstlisting}[language=Java]
public static MDDrawingView newSubjectRoleDrawingView(
	DrawingEditor editor, int width, int height) {

	Schema drawingViewSchema = SchemaLoader.load(
			schemaFactory,
			JHotDrawPrimitives.class, MDStandardDrawingView.class);

	FigureSelectionListenerSubjectRoleDataManager subjectRoleFactory =
		new FigureSelectionListenerSubjectRoleDataManager();

	DrawingViewSchemaFactory drawingViewSchemaFactory = subjectRoleFactory.factory(
			DrawingViewSchemaFactory.class, drawingViewSchema);

	MDStandardDrawingView drawingView = drawingViewSchemaFactory.DrawingView();

	MyJPanel jPanel = new MyJPanel();
	jPanel.setAutoscrolls(true);
	....
	drawingView.editor(editor);

	drawingView.size(new Dimension(width, height));
	jPanel.setSize(width, height);

	...

	// Panel events
	jPanel.addMouseListener(new MouseAdapter() {...});
	jPanel.addMouseMotionListener(new MouseMotionListener() {...});
	jPanel.addKeyListener(new DrawingViewKeyListener(drawingView));

	return drawingView;
}
	\end{lstlisting}
	\caption{ManagedDataJHotDraw: MDDrawingView Factory}
	\label{lst:ManagedDataJHotDraw_MDDrawingView_Factory}
\end{sourcecode}