% !TEX root = ../thesis.tex

%%%%%%%%%%%%%%%%%%%%%%%%%%%%%%%%%%%%%%%%%%%%%%%%%%%%%%%%%%%%%%%%%%%%%%%%%%%%%%%
% Chapter: Schema Loading
%%%%%%%%%%%%%%%%%%%%%%%%%%%%%%%%%%%%%%%%%%%%%%%%%%%%%%%%%%%%%%%%%%%%%%%%%%%%%%%
\chapter{Schema Loading}\label{appdx:SchemaLoading}

\section{Load method}

\begin{sourcecode} [H]
	\begin{lstlisting}[language=Java, escapechar=|]
public static Schema load(
	SchemaFactory factory, Schema schemaSchema, Class<?>... schemaKlassesDef) {

	// Filter out primitives by loading them separately
	final List<Class<?>> schemaKlasses = new LinkedList<>();
	for (Class<?> schemaClass : schemaKlassesDef) {
		if (Primitives.class.isAssignableFrom(schemaClass)) {
			primitiveManager.loadPrimitives(schemaClass);
		} else {
			schemaKlasses.add(schemaClass);
		}
	}

	// create an empty schema using the factory, will wire it later
	final Schema schema = factory.Schema();

	// build the types from the schema klasses definition
	final Set<Type> types = 
		buildTypesFromClasses(factory, schema, schemaSchema, schemaKlasses);

	// wire the types on schema
	// it is inverse so it will refer to schema.types() directly
	types.forEach(type -> type.schema(schema));

	// get the schema's schemaKlass
	final Klass schemaSchemaKlass = schemaSchema.klasses().stream()
		.filter(klass -> klass.name().equals("Schema"))
		.findFirst().orElse(null);

	// wire the schema's schemaKlass
	schema.schemaKlass(schemaSchemaKlass);
	return schema;
}
	\end{lstlisting}
	\caption{SchemaLoader load method}
	\label{lst:SchemaLoader_load}
\end{sourcecode}

\section{Build Types Method}

\begin{sourcecode} [H]
	\begin{lstlisting}[language=Java, escapechar=|]
private static Set<Type> buildTypesFromClasses(
	SchemaFactory factory,
	Schema schema,
	Schema schemaSchema,
	List<Class<?>> schemaKlassesDefinition) {
	Map<Type, TypeWithClass> types = new LinkedHashMap<>();

	// <classNameFieldNameCombo, FieldWithMethod>
	Map<String, FieldWithMethod> allFieldsWithReturnType = new LinkedHashMap<>();

	// Klasses
	for (Class<?> schemaKlassDefinition : schemaKlassesDefinition) {
		String klassName = schemaKlassDefinition.getSimpleName();

		Map<String, Field> fieldsForKlass =
			buildFieldsFromMethods(
				klassName, factory, schemaKlassDefinition, allFieldsWithReturnType);

		// create a new klass
		Klass klass = factory.Klass();
		klass.name(klassName);
		klass.schema(schema);

		// wire the owner klass in fields, it is inverse refering to klass.fields()
		fieldsForKlass.values().forEach(field -> field.owner(klass));

		typesCache.put(klass.name(), klass);

		// add the a new klass
		types.put(klass, new TypeWithClass(klass, schemaKlassDefinition));
	}

	// wiring
	wireFieldTypes(factory, schema, allFieldsWithReturnType);
	wireFieldInverse(allFieldsWithReturnType);
	wireFieldTypeKeys(types);

	wireSchemaKlasses(schemaSchema);

	wireKlassSupers(types, typesCache);
	wireKlassSubs(types, typesCache);
	wireKlassClassOf(types, schemaKlassesDefinition);

	typesCache.clear();

	return types.keySet();
}
    \end{lstlisting}
	\caption{SchemaLoader buildTypesFromClasses method}
	\label{lst:SchemaLoader_buildTypesFromClasses}
\end{sourcecode}

\section{Build Fields Method}

\begin{sourcecode} [H]
	\begin{lstlisting}[language=Java, escapechar=|]
public static Map<String, Field> buildFieldsFromMethods(
	String klassName,
	SchemaFactory factory,
	Class<?> schemaKlassDefinition,
	Map<String, FieldWithMethod> allFieldsWithReturnType) {
	Map<String, Field> fieldsForKlass = new LinkedHashMap<>();

	final Method[] fields = schemaKlassDefinition.getMethods();
	// sort methods
	...

	for (Method schemaKlassField : fields) {
		final String fieldName = schemaKlassField.getName();
		final Class<?> fieldReturnClass = schemaKlassField.getReturnType();

		// skip default method declarations
		if (schemaKlassField.isDefault()) {
			continue;
		}

		// check for many
		final boolean many = primitiveManager.isMany(fieldReturnClass);

		// check for optional
		final boolean optional = schemaKlassField.isAnnotationPresent(Optional.class);

		// check for key
		final boolean key = schemaKlassField.isAnnotationPresent(Key.class);

		// check for contain
		final boolean contain = schemaKlassField.isAnnotationPresent(Contain.class);

		// add its fields, the owner Klass will be added later
		final Field field = factory.Field();
		field.name(fieldName);
		field.contain(contain);
		field.key(key);
		field.many(many);
		field.optional(optional);

		fieldsForKlass.put(fieldName, field);

		// use klassName and fieldName combo here, 
		// because the real hashCode can not be calculated.
		allFieldsWithReturnType.put(
			klassName + fieldName, new FieldWithMethod(field, schemaKlassField));
	}
	return fieldsForKlass;
}
    \end{lstlisting}
	\caption{SchemaLoader buildFieldsFromMethods method}
	\label{lst:SchemaLoader_buildFieldsFromMethods}
\end{sourcecode}

\section{Wire Types Method}

\begin{sourcecode} [H]
	\begin{lstlisting}[language=Java, escapechar=|]
private static void wireFieldTypes(
	SchemaFactory factory,
	Schema schema,
	Map<String, FieldWithMethod> allFieldsWithReturnType) {

	for (String klassNameFieldNameCombo : allFieldsWithReturnType.keySet()) {
		Method method = allFieldsWithReturnType.get(klassNameFieldNameCombo).method;
		Field field = allFieldsWithReturnType.get(klassNameFieldNameCombo).field;

		// In case the field is multi value (many), that means that the real type is
		// not given in the method.getReturnType() because this will give Set ot List,
		// BUT the real type is in method.getGenericReturnType().
		Class<?> fieldTypeClass;

		// in case it is multi field, get the return the Generic Return Type
		if (field.many()) {
			// The type in this case will be Set or List,
			// but the Generic Return Type will be the actual type.
			ParameterizedType fieldManyType = 
				(ParameterizedType) method.getGenericReturnType();
			fieldTypeClass = (Class<?>) fieldManyType.getActualTypeArguments()[0];
		} else {
			fieldTypeClass = method.getReturnType();
		}

		Type fieldType = getFieldType(fieldTypeClass, schema, factory);
		field.type(fieldType);
	}
}
    \end{lstlisting}
	\caption{SchemaLoader wireFieldTypes method}
	\label{lst:SchemaLoader_wireFieldTypes}
\end{sourcecode}

Listing \ref{lst:SchemaLoader_load} demonstrates the loading process in managed data.
As illustrated we first implement the instances and following that we use setters to wire them up. 
The reason for this is that not everything exists at the time that it needs to be set.
Listing \ref{lst:SchemaLoader_buildTypesFromClasses} show how the klass creation is performed during the schema loading.
Listing \ref{lst:SchemaLoader_buildFieldsFromMethods} illustrates how the fields of a klass are created reflection.
Listing \ref{lst:SchemaLoader_wireFieldTypes} presents the way the types are wired during the schema creation.

This implementation shows the usage of Java reflection in our implementation.
However, because Java reflection capabilities are limited, this restricted our implementation.
