%%%%%%%%%%%%%%%%%%%%%%%%%%%%%%%%%%%%%%%%%%%%%%%%%%%%%%%%%%%%%%%%%%%%%%%%%%%%%%%
% Chapter: Introduction
%%%%%%%%%%%%%%%%%%%%%%%%%%%%%%%%%%%%%%%%%%%%%%%%%%%%%%%%%%%%%%%%%%%%%%%%%%%%%%%
\chapter{Introduction}\label{Introduction}
\ac{ccc} is a problem for which the classic programming techniques can not tackle with sufficiently. 
This results in scattered and tangled code, which affects the system's modularity and it's ease of maintenance and evolution. 
Since \ac{oop} and \ac{pp} techniques can not solve this problem, \ac{aop} presented \cite{kiczales1997aspect} in order to
provide a solution by introducing the notion of \textit{aspects}.

\ac{aop} results in a modular and \textit{single-responsibility} design whose properties must be implemented as \textit{components} (cleanly encapsulated procedure) and \textit{aspects} (not clearly encapsulated procedure), both separate concepts that are combined for the result through a process called \textit{weaving}. 
However, relying on \ac{aop}, paradoxically, does not improve the evolution of a project even with the modularity that it provides 
since it introduces tight coupling between the aspects and the application. 
As a result the way to tackle with this problem we need a more sophisticated and expressing crosscut language.
Consequently, \ac{ccc} could be handled in a higher level of the language such as the data structuring and management mechanisms.

Managed data \cite{loh2012managed} allows programmers to take control of important aspects of data as reusable modules. 
Using managed data a developer can build \textit{data managers} that handle the fundamental data manipulation primitives 
that are usually hard-coded in the programming language, by introducing custom data manipulation mechanisms. 
Managed data have been researched and implemented under the Enso project\footnote{\label{note1}\url{http://enso-lang.org/}}, which is developed in Ruby\footnote{\label{note2}\url{https://www.ruby-lang.org/en/}} (a dynamic programming language) using Ruby’s reflection capabilities. 
Furthermore, managed data are considered less able to be supported in static languages directly which makes it more challenging for 
this thesis since it is going to be implemented in Java.
In this thesis I am going to use the Java reflection capabilities to implement managed data and focus on specific aspects and design patterns implementations using the data managers concept of managed data. 

%%%%%%%%%%%%%%%%%%%%%%%%%%%%%%%%%%%%%%%%%%%%%%%%%%%%%%%%%%%%%%%%%%%%%%%%%%%%%%%
\section{Initial Study}\label{Initial Study}
% TODO: more
In their study on managed data, Loh et al. \cite{loh2012managed} present an implementation of managed data
in Ruby and they use as a case study a web development framework from the Enso project to reuse database management and 
access control mechanisms across different data definitions.

This thesis is an extension of their work; we implement managed data in Java (a static programming language) using the Java reflection API\footnote{\url{https://docs.oracle.com/javase/tutorial/reflect/}} and dynamic proxies\footnote{\url{https://docs.oracle.com/javase/8/docs/api/java/lang/reflect/Proxy.html}}. 
Although proxies in static programming languages can not implement the full range of managed data \cite{loh2012managed}. 
Java provides a strong implementation of the meta-object protocol \cite{kiczales1991art}, which can be used though the Java Reflection API \cite{forman2004java}. 
Additionally, this project will focus on aspects and will provide a solution to the \ac{ccc} problem by using managed data.

%%%%%%%%%%%%%%%%%%%%%%%%%%%%%%%%%%%%%%%%%%%%%%%%%%%%%%%%%%%%%%%%%%%%%%%%%%%%%%%
\section{Problem statement}\label{Problem statement}
The problem we study regards the \ac{ccc} that are scattered around the application, resulting to a hard to maintain system 
by tangling implementation logic and concerns code together.
Even though, \ac{aop} provides new modularization mechanisms, which should result in easier evolving software, 
it delivers solutions that are as hard and sometimes even harder to evolve than before \cite{tourwe2003existence}. 
The problem lays on the aspects, which have to include a crosscut description of all places in the application where this code yields an influence. 
Thus, the aspects are tightly coupled to the application and this greatly affects the evolvability of the overall system. 

Additionally, Friedrich Steimann \cite{steimann2005domain} argues that modeling languages are not aspect ready. 
The problem that arises is located at the level of software modeling. 
More specifically, in \textit{roles modeling}, whereas in \ac{oop} roles are tied to the collaborations,
collaborations rely on interactions of objects, and aspects on the other hand are typically defined independently of one another. 

Furthermore, in terms of order, it has been observed that aspects are not elements of the domain, they describe the order rather than the domain. 
Finally, aspects invariably express non-functional requirements, but if the non-functional requirements are not elements of domain models then neither are aspects.

In order to solve the aforementioned problems, we implement manage data, lifting the data management up to the application.

%%%%%%%%%%%%%%%%%%%%%%%%%%%%%%%%%%%%%%%%%%%%%%%%%%%%%%%%%%%%%%%%%%%%%%%%%%%%%%%
\subsection{Research Questions}\label{Research Questions}
Managed data has not been practically implemented in a static language before, therefore my first research questions states 
``Can managed Data be implemented in a static language like Java?''. 
Based in the previous argumentation about the relevance of \ac{aop} and the solutions that managed data can provide in \ac{ccc}, my second research question is ``Can managed data solve \ac{ccc} and to what extend does it improve the software evolution problems that \ac{aop} introduces in a modular solution?''. 
Finally by using a software showcase, the JHotDraw framework, as well as its \ac{aop} implementation AJHotDraw \cite{marinajhotdraw}, 
I am going to evaluate the implementation of managed data on an inventory of aspects and design patterns. 
As a result the third research question states ``To what extent can managed data tame an inventory of aspects and design patterns in the JHotDraw framework, in contrast with the original and the AOP implementation.''

%%%%%%%%%%%%%%%%%%%%%%%%%%%%%%%%%%%%%%%%%%%%%%%%%%%%%%%%%%%%%%%%%%%%%%%%%%%%%%%
\subsection{Solution Outline}\label{Solution Outline}
Our solution is consisted of an implementation of managed data in Java. 
This framework can be used by applications in order to deal with \ac{ccc}.

To validate our hypotheses we are going to implement managed data in Java using the Java Reflection API and Dynamic Proxies. 
More specifically we are going to use Java interfaces for \textit{schemas} and dynamic proxies for \textit{data managers}. 
Furthermore, we are going to provide as a proof of concept the examples given in \cite{loh2012managed} but this time developed in Java. As mentioned in \cite{loh2012managed} to stack data managers I am going to use the Decorator Pattern \cite{gamma1995design}. 

In order to prove that managed data solves the problems that \ac{aop} introduces, we are going to implement an inventory of the following aspects and design patterns from JHotDraw using data managers:
\begin{description}
  \item[The Observer Pattern], which as presented in literature \cite{tourwe2003existence} \cite{hannemann2005role} \cite{marin2005approach}, is by nature not modularized and the scatters pattern code through the classes. 
  This pattern is considered as a difficult case because it is used a lot in the original JHotDraw source code but with multiple variations, thus it is difficult to extract an abstract version.

  \item[The Singleton Pattern], which as presented \cite{hannemann2005role} \cite{hannemann2002design}, it can easily be abstracted as an aspect and replace the \ac{oop} usage in JHotDraw. 

  \item[The Template Method], which as presented \cite{hannemann2005role} \cite{hannemann2002design}, it scatters code by introducing roles such as those of \textit{AbstractClass} and \textit{ConcreteClass}.

  \item[The Undo aspect], which is analyzed extensively \cite{marin2004refactoring} and a solution is provided by AJHotDraw. 
  More specifically, this aspect consists of aspect-oriented refactoring of the \textit{Command} pattern with \textit{Undo} actions.
\end{description}

This inventory is implemented using data managers that have modularity as a main characteristic and is been evaluated in a new JHotDraw implementation. 
We compare those aspects with the original version of JHotDraw, and the aspect version, AJHotDraw. 
Since our solution is a refactoring of the JHotDraw framework we need a way to ensure the behavioral equivalence between the original and the refactored solution \cite{fowler2009refactoring}. 
However, JHotDraw comes with no tests. 
Thus, we use the TestJHotDraw, which is a subproject of the AJHotDraw development team, and it is developed in order to contribute to a gradual and safe adoption of aspect-oriented techniques in existing applications and allow for a better assessment of aspect orientation.
Since we use our JHotDraw implementation for the functional evaluation of our solution, we can use the presented criteria \cite{hannemann2002design}, which are \textit{Locality},\textit{Re-usability}, \textit{Composition Transparency}, and \textit{(Un)pluggability}, in order to present metrics of our solution. 

%%%%%%%%%%%%%%%%%%%%%%%%%%%%%%%%%%%%%%%%%%%%%%%%%%%%%%%%%%%%%%%%%%%%%%%%%%%%%%%
\subsection{Research Method}\label{Research Method}
The answers for the research questions will be extracted from the background material, the managed data implementation and the comparison of metrics gathered by the evaluation of the implementation. 
% TODO 

%%%%%%%%%%%%%%%%%%%%%%%%%%%%%%%%%%%%%%%%%%%%%%%%%%%%%%%%%%%%%%%%%%%%%%%%%%%%%%%
\section{Contributions}\label{Contributions}

\begin{description}
  \item[Contribution 1: Managed data implementation in Java]

  \item[Contribution 2: Managed data Java framework]
  The final deliverable is a Java library with which the developer can define and implement aspects as reusable modules
and integrate them with an application without mixing the business logic with concern logic.
More specifically, the schemas and the data managers have to be defined by the developer, as well as any additional
functionality that may needed to be integrated to the patterns or roles of the application.

  \item[Contribution 3: Managed data Evaluation in JHotDraw]

  \item[Contribution 4: JHotDraw implementation results assessment and comparison with AJHotDraw]

\end{description}

%%%%%%%%%%%%%%%%%%%%%%%%%%%%%%%%%%%%%%%%%%%%%%%%%%%%%%%%%%%%%%%%%%%%%%%%%%%%%%%
\section{Related Work}\label{Related Work}

%%%%%%%%%%%%%%%%%%%%%%%%%%%%%%%%%%%%%%%%%%%%%%%%%%%%%%%%%%%%%%%%%%%%%%%%%%%%%%%
\section{Document Outline}\label{Document Outline}
