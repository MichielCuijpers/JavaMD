%%%%%%%%%%%%%%%%%%%%%%%%%%%%%%%%%%%%%%%%%%%%%%%%%%%%%%%%%%%%%%%%%%%%%%%%%%%%%%%
% Chapter: Introduction
%%%%%%%%%%%%%%%%%%%%%%%%%%%%%%%%%%%%%%%%%%%%%%%%%%%%%%%%%%%%%%%%%%%%%%%%%%%%%%%
\chapter{Introduction}\label{Introduction}
\ac{ccc} is a problem for which the classic programming techniques can not tackle with sufficiently. 
This results in scattered and tangled code, which affects the system's modularity and it's ease of maintenance and evolution. 
Since \ac{oop} and \ac{pp} techniques can not solve this problem, \ac{aop} presented \cite{kiczales1997aspect} in order to
provide a solution by introducing the notion of \textit{aspects}.

\ac{aop} results in a modular and \textit{single-responsibility} design whose properties must be implemented as \textit{components} (cleanly encapsulated procedure) and \textit{aspects} (not clearly encapsulated procedure), both separate concepts that are combined for the result through a process called \textit{weaving}. 
However, relying on \ac{aop}, paradoxically, does not improve the evolution of a project even with the modularity that it provides 
since it introduces tight coupling between the aspects and the application. 
As a result the way to tackle with this problem we need a more sophisticated and expressing crosscut language.
Consequently, \ac{ccc} could be handled in a higher level of the language such as the data structuring and management mechanisms.

Managed data \cite{loh2012managed} allows programmers to take control of important aspects of data as reusable modules. 
Using managed data a developer can build \textit{data managers} that handle the fundamental data manipulation primitives 
that are usually hard-coded in the programming language, by introducing custom data manipulation mechanisms. 
Managed data have been researched and implemented under the Enso project\footnote{\label{note1}\url{http://enso-lang.org/}}, which is developed in Ruby\footnote{\label{note2}\url{https://www.ruby-lang.org/en/}} (a dynamic programming language) using Ruby’s reflection capabilities. 
Furthermore, managed data are considered less able to be supported in static languages directly which makes it more challenging for 
this thesis since it is going to be implemented in Java.
In this thesis I am going to use the Java reflection capabilities to implement managed data and focus on specific aspects and design patterns implementations using the data managers concept of managed data. 

%%%%%%%%%%%%%%%%%%%%%%%%%%%%%%%%%%%%%%%%%%%%%%%%%%%%%%%%%%%%%%%%%%%%%%%%%%%%%%%
\section{Initial Study}\label{Initial Study}
In their study on managed data, A Loh et al. \cite{loh2012managed} present an implementation of managed data
in Ruby and they use as a case study a web development framework from the Enso project to reuse database management and 
access control mechanisms across different data definitions.

% TODO: add more details here

This thesis is support and an extension of their work; we implement managed data in Java (a static programming language) using the Java reflection API\footnote{\url{https://docs.oracle.com/javase/tutorial/reflect/}} and dynamic proxies\footnote{\url{https://docs.oracle.com/javase/8/docs/api/java/lang/reflect/Proxy.html}}. 
Although proxies in static programming languages can not implement the full range of managed data \cite{loh2012managed}. 
Java provides a strong implementation of the meta-object protocol \cite{kiczales1991art}, which can be used though the Java Reflection API \cite{forman2004java}. 
Additionally, this project will focus on aspects and will provide a solution to the \ac{ccc} problem by using managed data.

%%%%%%%%%%%%%%%%%%%%%%%%%%%%%%%%%%%%%%%%%%%%%%%%%%%%%%%%%%%%%%%%%%%%%%%%%%%%%%%
\section{Problem statement}\label{Problem statement}
% Cross cutting concerns
% AOP problems

%%%%%%%%%%%%%%%%%%%%%%%%%%%%%%%%%%%%%%%%%%%%%%%%%%%%%%%%%%%%%%%%%%%%%%%%%%%%%%%
\subsection{Research Questions}\label{Research Questions}

%%%%%%%%%%%%%%%%%%%%%%%%%%%%%%%%%%%%%%%%%%%%%%%%%%%%%%%%%%%%%%%%%%%%%%%%%%%%%%%
\subsection{Solutions Outline}\label{Solutions Outline}

%%%%%%%%%%%%%%%%%%%%%%%%%%%%%%%%%%%%%%%%%%%%%%%%%%%%%%%%%%%%%%%%%%%%%%%%%%%%%%%
\subsection{Research Method}\label{Research Method}

%%%%%%%%%%%%%%%%%%%%%%%%%%%%%%%%%%%%%%%%%%%%%%%%%%%%%%%%%%%%%%%%%%%%%%%%%%%%%%%
\section{Contributions}\label{Contributions}

%%%%%%%%%%%%%%%%%%%%%%%%%%%%%%%%%%%%%%%%%%%%%%%%%%%%%%%%%%%%%%%%%%%%%%%%%%%%%%%
\section{Related Work}\label{Related Work}

%%%%%%%%%%%%%%%%%%%%%%%%%%%%%%%%%%%%%%%%%%%%%%%%%%%%%%%%%%%%%%%%%%%%%%%%%%%%%%%
\section{Document Outline}\label{Document Outline}
