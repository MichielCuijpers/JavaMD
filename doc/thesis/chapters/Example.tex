% !TEX root = ../thesis.tex

%%%%%%%%%%%%%%%%%%%%%%%%%%%%%%%%%%%%%%%%%%%%%%%%%%%%%%%%%%%%%%%%%%%%%%%%%%%%%%%
% Chapter: Example Applications
%%%%%%%%%%%%%%%%%%%%%%%%%%%%%%%%%%%%%%%%%%%%%%%%%%%%%%%%%%%%%%%%%%%%%%%%%%%%%%%
\chapter{Example Application: State Machine Monitoring}\label{Example Application}
In order to show the contribution of our implementation in terms aspect refactoring, we are going to present a simple example.
The example consists of a simple state machines application, which is inspired by Martin Fowler \cite{fowler2010domain}.

% TODO: Rephrase
The requirements of the application are simple: A state machine consists of a number of named state declarations, where each state contains transitions to other states (identified by name) when a certain event happens.

In our example we want to first define these models, next interpret the definition given a list of event and finally log any changed in the console.

%%%%%%%%%%%%%%%%%%%%%%%%%%%%%%%%%%%%%%%%%%%%%%%%%%%%%%%%%%%%%%%%%%%%%%%%%%%%%%%
\section{The Object Oriented Way}

%%%%%%%%%%%%%%%%%%%%%%%%%%%%%%%%%%%%%%%%%%%%%%%%%%%%%%%%%%%%%%%%%%%%%%%%%%%%%%%
\subsection{Models definition}

%%%%%%%%%%%%%%%%%%%%%%%%%%%%%%%%%%%%%%%%%%%%%%%%%%%%%%%%%%%%%%%%%%%%%%%%%%%%%%%
\subsection{A simple program}

%%%%%%%%%%%%%%%%%%%%%%%%%%%%%%%%%%%%%%%%%%%%%%%%%%%%%%%%%%%%%%%%%%%%%%%%%%%%%%%
\subsection{Monitor concern}

%%%%%%%%%%%%%%%%%%%%%%%%%%%%%%%%%%%%%%%%%%%%%%%%%%%%%%%%%%%%%%%%%%%%%%%%%%%%%%%
\subsubsection{A monitoring program}

%%%%%%%%%%%%%%%%%%%%%%%%%%%%%%%%%%%%%%%%%%%%%%%%%%%%%%%%%%%%%%%%%%%%%%%%%%%%%%%
\section{The Managed Data Way}

%%%%%%%%%%%%%%%%%%%%%%%%%%%%%%%%%%%%%%%%%%%%%%%%%%%%%%%%%%%%%%%%%%%%%%%%%%%%%%%
\subsection{Schemas definition}
First we need to define all the models of the state machine application. 
As extracted from the requirements, we need a \texttt{Machine} \ref{lst:Machine_Schema}, a \texttt{State} \ref{lst:State_Schema}, and a \texttt{Transition} \ref{lst:Transition_Schema} between states.

%%%%%%%%%%%%%%%%%%%%%%%%%%%%%%%%%%%%%%%%%%%%%%%%%%%%%%%%%%%%%%%%%%%%%%%%%%%%%%%
\begin{sourcecode}
	\begin{lstlisting}[language=Java,escapechar=|]
public interface Machine {
	State start(State... state);

	@Contain
	Set<State> states(State... state);
}
	\end{lstlisting}
	\caption{The Machine Schema}
	\label{lst:Machine_Schema}
\end{sourcecode}

In the \texttt{Machine} schema definition, listing \ref{lst:Machine_Schema} all we need is a \texttt{start}ing state and a set of \texttt{states} that the machine can be into at each time.
The \texttt{@Contain} annotation here defines that the \texttt{states} field of this schema is part of the spine tree, but more on this on the implementation Chapter \ref{Schema Definition}.

%%%%%%%%%%%%%%%%%%%%%%%%%%%%%%%%%%%%%%%%%%%%%%%%%%%%%%%%%%%%%%%%%%%%%%%%%%%%%%%
\begin{sourcecode}[H]
	\begin{lstlisting}[language=Java,escapechar=|]
public interface State extends M {
	@Key
	String name(String... name);

	@Inverse(other = Machine.class, field = "states")
	Machine machine(Machine... machine);

	@Contain
	Set<Transition> out(Transition... transition);

	@Contain
	Set<Transition> in(Transition... transition);
}
	\end{lstlisting}
	\caption{The State Schema}
	\label{lst:State_Schema}
\end{sourcecode}

For the \texttt{State} schema definition, listing \ref{lst:State_Schema}, we need a \texttt{name} field, which is representing name of the state. 
As it can be seen the \texttt{name} field has been annotated with the \texttt{@Key} annotation, which indicates that this field will be used as a key in case of a \texttt{Set} instance of this schema.
Additionally, the schema includes a set of \texttt{in} and \texttt{out} transitions.
Since those two fields are of type \texttt{Set}, that means that a field of the \texttt{Transition} schema has to be marked as key.
In this case is the name (line \ref{line:transition_key} listing \ref{lst:Transition_Schema}).
Finally, the field \texttt{machine} represents which state machine's is the current state part of. 
As it can be seen in the schema definition, listing \ref{lst:State_Schema}, the \texttt{machine} field has been annotated with \texttt{@Inverse}, which indicates that this field is a reference to a field of an other schema, in this case this schema is the \texttt{Machine} schema and the field is the \texttt{states} field.

%%%%%%%%%%%%%%%%%%%%%%%%%%%%%%%%%%%%%%%%%%%%%%%%%%%%%%%%%%%%%%%%%%%%%%%%%%%%%%%
\begin{sourcecode}[H]
	\begin{lstlisting}[language=Java,escapechar=|]
public interface Transition extends M {
	@Key 					|\label{line:transition_key}| 
	String event(String... event);

	@Inverse(other = State.class, field = "out")
	State from(State... from);

	@Inverse(other = State.class, field = "in")
	State to(State... to);
}
	\end{lstlisting}
	\caption{The Transition Schema}
	\label{lst:Transition_Schema}
\end{sourcecode}

Finally, for the \texttt{Transition} schema definition, listing \ref{lst:Transition_Schema}, all we need is an \texttt{event} which represents the event of the transition and it is also the \textbf{key}.
Additionally, the \texttt{from} and \texttt{to} states represent the state that the machine changes from and to respectively.
However, those are just reference to the \texttt{State} schema, listing \ref{lst:State_Schema}, to the \texttt{in} and \texttt{out} fields respectively, since they are defined with the \texttt{@Inverse} annotation.

%%%%%%%%%%%%%%%%%%%%%%%%%%%%%%%%%%%%%%%%%%%%%%%%%%%%%%%%%%%%%%%%%%%%%%%%%%%%%%%
\subsection{Factory definition}
The question now is how can we build instances of managed objects for those schemas. For Java to create the three schemas as managed data, we need to define a factory, which will be the one that creates managed data instances for each of these the defined schemas \ref{lst:StateMachineFactory}.

\begin{sourcecode}[H]
	\begin{lstlisting}[language=Java]
public interface StateMachineFactory {
    State State();
    Machine Machine();
    Transition Transition();
}
	\end{lstlisting}
	\caption{The StateMachine Factory}
	\label{lst:StateMachineFactory}
\end{sourcecode}

%%%%%%%%%%%%%%%%%%%%%%%%%%%%%%%%%%%%%%%%%%%%%%%%%%%%%%%%%%%%%%%%%%%%%%%%%%%%%%%
\subsection{Data managers}

%%%%%%%%%%%%%%%%%%%%%%%%%%%%%%%%%%%%%%%%%%%%%%%%%%%%%%%%%%%%%%%%%%%%%%%%%%%%%%%
\subsubsection{Basic Data Manager}

%%%%%%%%%%%%%%%%%%%%%%%%%%%%%%%%%%%%%%%%%%%%%%%%%%%%%%%%%%%%%%%%%%%%%%%%%%%%%%%
\subsubsection{A simple program}

%%%%%%%%%%%%%%%%%%%%%%%%%%%%%%%%%%%%%%%%%%%%%%%%%%%%%%%%%%%%%%%%%%%%%%%%%%%%%%%
\subsection{Monitoring concern}

%%%%%%%%%%%%%%%%%%%%%%%%%%%%%%%%%%%%%%%%%%%%%%%%%%%%%%%%%%%%%%%%%%%%%%%%%%%%%%%
\subsubsection{Observable Data Manager}

%%%%%%%%%%%%%%%%%%%%%%%%%%%%%%%%%%%%%%%%%%%%%%%%%%%%%%%%%%%%%%%%%%%%%%%%%%%%%%%
\subsubsection{A monitoring program}

% %%%%%%%%%%%%%%%%%%%%%%%%%%%%%%%%%%%%%%%%%%%%%%%%%%%%%%%%%%%%%%%%%%%%%%%%%%%%%%%
% \section{Immutability}

% %%%%%%%%%%%%%%%%%%%%%%%%%%%%%%%%%%%%%%%%%%%%%%%%%%%%%%%%%%%%%%%%%%%%%%%%%%%%%%%
% \subsection{Lockable Data Manager}

% %%%%%%%%%%%%%%%%%%%%%%%%%%%%%%%%%%%%%%%%%%%%%%%%%%%%%%%%%%%%%%%%%%%%%%%%%%%%%%%
% \subsubsection{Execution}

