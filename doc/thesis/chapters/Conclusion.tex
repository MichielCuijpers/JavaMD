% !TEX root = ../thesis.tex

%%%%%%%%%%%%%%%%%%%%%%%%%%%%%%%%%%%%%%%%%%%%%%%%%%%%%%%%%%%%%%%%%%%%%%%%%%%%%%%
% Chapter: Conclusion
%%%%%%%%%%%%%%%%%%%%%%%%%%%%%%%%%%%%%%%%%%%%%%%%%%%%%%%%%%%%%%%%%%%%%%%%%%%%%%%
\chapter{Conclusion}\label{Conclusion}
% TODO:
% In this research we have presented a managed data implementation in Java using its reflection API and dynamic proxies.
% By doing that we allow the programmer to take control over the mechanisms of data creation and manipulation.
% We defined our language using Java's syntax by using interfaces and annotations.

% Having the managed data implementation in place, we refactored an existing Object Oriented use case, the JHotDraw.
% This use case is considered as a well-designed \ac{oop} system; however, it includes the problem of the \ac{ccc}.
% Thus, we refactored this system using managed data in order to solve the \ac{ccc} problem.
% More specifically, we have migrated some main components of JHotDraw to managed data, then we removed the \ac{ccc} and finally we used data managers to implement them.
% This refactoring led to a new version of JHotDraw the ManagedDataJHotDraw which solves the problem of some main \ac{ccc} of the original application.

% During the assessment of our refactoring we have collected a number of metrics that we used in order to evaluate our refactoring in comparison with the original application.
% Moreover, we extensively presented the refactoring process of JHotDraw and compared with AJHotDraw, the Aspect Oriented implementation of JHotDraw.
% Finally, by presenting a set of metrics and a number of modularity properties, we assessed our results comparing them in with the original and the Aspect Oriented version.

% Overall, we showed that managed data can be implemented in a static language and it can tame aspects by using data managers for concern implementation.
% Moreover, it extends some capabilities of AspectJ and deviates from the problem of the coupling between aspect definition and components.
