% !TEX root = ../thesis.tex

%%%%%%%%%%%%%%%%%%%%%%%%%%%%%%%%%%%%%%%%%%%%%%%%%%%%%%%%%%%%%%%%%%%%%%%%%%%%%%%
% Chapter: Aspect refactoring 
%%%%%%%%%%%%%%%%%%%%%%%%%%%%%%%%%%%%%%%%%%%%%%%%%%%%%%%%%%%%%%%%%%%%%%%%%%%%%%%
\chapter{Aspect refactoring of JHotDraw in managed data}\label{AspectRefactoring}

\section{Introduction}

\section{Aspect Refactoring}

\subsection{Aspect Solution Templates}

\section{Aspect Refactoring of JHotDraw}

\subsection{Command Pattern in JHotDraw}\label{Command Pattern in JHotDraw}

\subsection{Undo Concern of JHotDraw}\label{Undo JHotDraw}

\subsubsection{Current Undo Implementation}
% A number of activities in JHOTDRAW, such as handling font sizes and colors, image rotation, or inserting the clipboard's content into a drawing, support the undo functionality. 



\subsubsection{Command Pattern}

\subsection{The Observer Pattern in JHotDraw}\label{The Observer Pattern in JHotDraw}

% \ref{Observer pattern in Aspect Oriented Programming}
% \ref{Refactoring of ccc}

\section{Compare to AJHotDraw}