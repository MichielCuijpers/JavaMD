% !TEX root = ../thesis.tex

%%%%%%%%%%%%%%%%%%%%%%%%%%%%%%%%%%%%%%%%%%%%%%%%%%%%%%%%%%%%%%%%%%%%%%%%%%%%%%%
% Chapter: Aspect refactoring 
%%%%%%%%%%%%%%%%%%%%%%%%%%%%%%%%%%%%%%%%%%%%%%%%%%%%%%%%%%%%%%%%%%%%%%%%%%%%%%%
\chapter{Taming Aspects of JHotDraw with managed data}\label{AspectRefactoring}

To put our implementation and practice and evaluate the ability of managed data to tame aspects, we have refactor some of the \ac{ccc} of JHotDraw.
More specifically, in this chapter is presented the refactoring of the \textit{FigureSelectionListener} observer pattern as well as the \textit{Undo} concerns of JHotDraw.
For their refactoring we used our implementation of managed data in Java that presented in the previous chapter.
However, to do that, we have migrated JHotDraw in managed data.
The result of this migration is available to an open-source project called ManagedDataJHotDraw\footnote{\url{https://github.com/TheolZacharopoulos/ManagedDataJHotDraw}}.
Additionally, we claim that this is the first aspect refactoring of an application using managed data to date.

\section{Refactoring Process}
The refactoring of such application like JHotDraw required a significant amount of time to study and familiarizing with the existing system.
In particular, we tried to focus only on the parts that we were going to refactor since we wanted to assess the same refactorings that AJHotDraw developers \cite{marinajhotdraw} did.
Thanks to their fan-in analysis \cite{marin2004identifying}, with which identified a number of aspects in the legacy system, we also emphasize on them in order to make a fair comparison.
Furthermore, while the AJHotDraw focused on a completely new version of JHotDraw written using AspectJ, we implemented ManagedDataJHotDraw maintaining the coherence and the original design.

We first need to migrate ....

\section{Migration of JHotDraw to Managed Data}

Everything as managed data?

\subsection{The MDStandardDrawingView}
% Add the dependencies as primitives
% Change every Drawing view to MDStdDrawingView
% Can not extend JPanel 
% MyJPanel wrapper
% Wrapper for JPanel, we need it for having access into the MDStandardDrawingView

\subsubsection{MDStandardDrawingView Factory}
% Initialization

\subsection{Issues}
% Pure managed data

\subsection{Limitations}
% Public access to all methods (default)
% Not private class
% Transient
% Synchronized
% Can use annotations for that though???

\section{Aspect Refactoring of JHotDraw}
Aspect refactoring refers to the refactoring of legacy in aspect oriented code. 
However, here we present an aspect refactoring of legacy code in managed data.

\subsection{FigureSelectionListener}
The \texttt{FigureSelectionListener} observer pattern of JHotDraw is a case that first presented by Hannemann et al. \cite{hannemann2005role} in their tole-based refactoring of design patterns in AspectJ. 
Later Marin et al. picked the same aspect and migrated it to their AJHotDraw implementation.
Likewise, we have also considered the same aspect for our refactoring in order to assess our concern solution with the existing one.

%%%%%%%%%%%%%%%%%%%%%%%%%%%%%%%%%%%%%%%%%%%%%%%%%%%%%%%%%%%%%%%%%%%%%%%%%%%%%%%
\subsection{FigureSelectionListener in JHotDraw}
The original observer pattern of the \texttt{FigureSelectionListener} functionality in JHotDraw is illustrated in Figure \ref{fig:JHotDraw_FigureSelectionListener_OOP}.

\begin{figure}[H]
	\centering
  	\fbox{\includegraphics[width=1\textwidth]{figures/JHotDraw_FigureSelectionListener_OOP.png}}
  	\caption{FigureSelectionListener in JHotDraw}
  	\label{fig:JHotDraw_FigureSelectionListener_OOP}
\end{figure}

As the figure illustrates, 

%%%%%%%%%%%%%%%%%%%%%%%%%%%%%%%%%%%%%%%%%%%%%%%%%%%%%%%%%%%%%%%%%%%%%%%%%%%%%%%
\subsection{Refactoring FigureSelectionListener in AJHotDraw}

\begin{figure}[H]
	\centering
  	\fbox{\includegraphics[width=.8\textwidth]{figures/JHotDraw_FigureSelectionListener_AOP.png}}
  	\caption{FigureSelectionListener in AJHotDraw}
  	\label{fig:JHotDraw_FigureSelectionListener_AOP}
\end{figure}

\subsubsection{Role Superimposition}

\subsubsection{Consistent Behavior}
\begin{sourcecode} [H]
	\begin{lstlisting}[language=AspectJ, escapechar=|]
public aspect SelectionChangedNotification {
	pointcut invalidateSelFigure(StandardDrawingView sdw) :
		(   withincode(boolean StandardDrawingView.addToSelectionImpl(Figure)) 
		 || withincode(void StandardDrawingView.removeFromSelection(Figure)))
		&& call(void Figure.invalidate()) 
		&& this(sdw);

	pointcut clear_toggleSelection(StandardDrawingView sdw):
		(execution(void StandardDrawingView.clearSelection()) ||
		 execution(void StandardDrawingView.toggleSelection(Figure)))
		&& this(sdw);

	after(StandardDrawingView sdw): invalidateSelFigure(sdw) {
		sdw.fireSelectionChanged();
	}

	after(StandardDrawingView sdw): clear_toggleSelection(sdw) {
		sdw.fireSelectionChanged();
	}
}
	\end{lstlisting}
	\caption{AJHotDraw: Consistent Behavior in FigureSelectionListener}
	\label{lst:Consistent Behavior in FigureSelectionListener}
\end{sourcecode}

%%%%%%%%%%%%%%%%%%%%%%%%%%%%%%%%%%%%%%%%%%%%%%%%%%%%%%%%%%%%%%%%%%%%%%%%%%%%%%%
\subsection{Refactoring FigureSelectionListener in ManagedDataJHotDraw}

\begin{figure}[H]
	\centering
  	\fbox{\includegraphics[width=1\textwidth]{figures/JHotDraw_FigureSelectionListener_MD.png}}
  	\caption{FigureSelectionListener in ManagedDataJHotDraw}
  	\label{fig:JHotDraw_FigureSelectionListener_MD}
\end{figure}

\subsection{SubjectRole Data Manager}
\subsubsection{Data manager}
\subsubsection{MObject}
\subsubsection{Predicate}
\subsubsection{Action}

\subsection{SubjectRole Integration}
\subsubsection{Consistent Behavior Predicate}
\subsubsection{FigureSelectionChanged Action}

% \subsection{Undo Concern}
% \subsubsection{Undo Concern in JHotDraw}
% \subsubsection{Refactoring Undo in AJHotDraw}
% \subsubsection{Refactoring Undo in ManagedDataJHotDraw}

\section{Claims}

\section{Threads to Validity}
% all study designs have flaws. By acknowledging them explicitly, the researchers show that they are aware of the flaws and have taken reasonable steps to minimize their effects.

% Construct Validity: Efficient intepretation of the results 
% Internal validity: Study design
% External validity: Justified results (is it the right case?)
% Reliability validity: same results on replication?